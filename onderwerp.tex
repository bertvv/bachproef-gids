\chapter{Een onderwerp kiezen}
\label{ch:onderwerp}

% Onderzoeksdomein
% Onderzoeksvraag + deelvragen
% Titel formuleren
%  - geen afkortingen, vakjargon
%  - lang genoeg, concreet
% Hoe vind je een onderwerp?
%
% ga op zoek naar de actualiteit in het domein dat je het meeste interesseert:
%  - Er bestaan verschillende portaalsites voor actuele ict-gerelateerde onderwerpen waar technische artikels, presentaties, interviews, enz. op verschijnen, bv. dzone.com, infoq.com, enz.
%  - Ga op zoek naar relevante conferenties (vb. Google IO, WWDC, http://lanyrd.com/topics/android/, \ldots). Vaak vind je video-opnamen van de presentaties op Youtube/Vimeo/\ldots
%  - Wie zijn de belangrijkste namen in de ``community''? Keynote-speakers op conferenties, auteurs van de belangrijkste boeken over het onderwerp, enz.
%  - Volg deze personen op Twitter, zoek uit of ze een blog hebben, actief zijn op LinkedIn, enz. Lees al wat je kan vinden dat ze geschreven hebben de laatste jaren.
%  - Newsletters, bv. Cron.Weekly, Devops Weekly
%
%  Dan zou je in principe de belangrijkste thema's moeten herkennen waar men op dit moment vooral mee bezig is en dat zou inspiratie kunnen geven voor je onderwerp.
%
% Niet speculeren over de toekomst.
%
% Voor jezelf brainstorm, mindmap maken. Raamwerk voor state-of-the-art
%
% Neem je tijd, bv. 2 weken elke dag 30 min/1u!

% Waaraan moet een onderwerp voldoen
% * Er is een concrete, duidelijk afgebakende onderzoeksvraag, onderzoeksdoelstelling
% * Voorstel is vernieuwend en heeft een duidelijke meerwaarde voor een specifieke doelgroep uit het ict-werkveld
% * De methodologie is duidelijk verantwoord, onderzoekstechnieken zijn geschikt voor beantwoorden onderzoeksvraag
% * Er is een duidelijke eigen bijdrage en technische diepgang

% Goede onderzoeksvragen
% - gaan niet over bestaande situatie (bv. Wat is data mining? Welke PHP frameworks zijn er?)
% - Op een onderzoeksvraag is nu nog geen antwoord
